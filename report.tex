\documentclass[a4paper]{article}
\usepackage[margin=2.5cm]{geometry}
\usepackage[portuguese]{babel}
\usepackage[utf8]{inputenc}
\usepackage[T1]{fontenc}
\usepackage{multirow}
\usepackage{siunitx}

\title{\LARGE \textbf{Relatório Inteligência Artificial}}
\author{Mihail Brinza \\ \scriptsize 83533 \normalsize \and Ricardo Brancas \\ \scriptsize 83557 \normalsize}

\begin{document}
    \maketitle

    \section{Resultados}

    \begin{table}[h!]
        \begin{tabular}{ c|*{3}{c}|*{3}{c}|*{3}{c} }
            & \multicolumn{3}{ c| }{DFS} & \multicolumn{3}{ c| }{Greedy} & \multicolumn{3}{ c }{A*} \\ \cline{2-10}
            Problema & t & g & e & t & g & e & t & g & e \\
            \hline
            1 & 0.0001 & 1 & 1 & 0.0002 & 2 & 1 & 0.0002 & 2 & 1     \\
            2 & 0.0004 & 5 & 4 & 0.0006 & 5 & 3 & 0.0006 & 6 & 4     \\
            3 & 5.8729 & 74702 & 74702 & 12.091 & 74703 & 74702 & 12.030 & 74703 & 74703 \\
            4 & 0.0052 & 55 & 54 & 0.0105 & 44 & 42 & 0.0058 & 26 & 24    \\
            5 & 221.30 & 3123309 & 3123308 & 0.0357 & 171 & 169 & 0.0145 & 18 & 16    \\
        \end{tabular}
        \caption{Resultados da execução dos vários problemas. \small \textbf{t} refere-se ao tempo demorado; \textbf{g} ao número de nós gerados e \textbf{e} ao número de nós expandidos.}
    \end{table}

    \section{Análise}

\end{document}
