\documentclass[a4paper]{article}
\usepackage[margin=3cm]{geometry}
\usepackage[portuguese]{babel}
\usepackage[utf8]{inputenc}
\usepackage[T1]{fontenc}
\usepackage{multirow}
\usepackage{amsmath}

\linespread{1.1}

\title{\LARGE \textbf{Relatório do Projeto 1 de Inteligência Artificial}}
\author{Mihail Brinza \\ \scriptsize 83533 \normalsize \and Ricardo Brancas \\ \scriptsize 83557 \normalsize}

\begin{document}
    \maketitle

    \section{Resultados Exeprimentais}

    \subsection{Breve descrição dos problemas utilizados}

    \subparagraph{Problema 1}
    [4x5], 2 cores, sem solução

    \subparagraph{Problema 2}
    [4x5], 3 cores

    \subparagraph{Problema 3}
    [10x4], 5 cores, sem solução

    \subparagraph{Problema 4}
    [10x4], 3 cores

    \subparagraph{Problema 5}
    [10x4], 5 cores

    \subsection{Resultados}

    \begin{table}[h!]
		\begin{center}
	        \begin{tabular}{ c|*{3}{c}|*{3}{c}|*{3}{c} }
	            & \multicolumn{3}{ c| }{DFS} & \multicolumn{3}{ c| }{Greedy} & \multicolumn{3}{ c }{A*} \\ \cline{2-10}
	            Problema & t & g & e & t & g & e & t & g & e \\
	            \hline
	            1 & 0.0003 & 0 & 1 & 0.0005 & 0 & 1 & 0.0006 & 0 & 1 \\
	            2 & 0.0006 & 7 & 4 & 0.0009 & 6 & 3 & 0.0009 & 7 & 4 \\
	            3 & 12.574 & 74701 & 74702 & 17.673 & 74701 & 74702 & 12.833 & 74701 & 74702 \\
	            4 & 0.0057 & 85 & 54 & 0.0144 & 59 & 42 & 0.0119 & 43 & 24 \\
	            5 & 208.09 & 3123363 & 3123308 & 0.0539 & 220 & 149 & 0.0306 & 91 & 16 \\
	        \end{tabular}
		\end{center}
		\footnotesize \textbf{t} refere-se ao tempo demorado; \textbf{g} ao número de nós gerados e \textbf{e} ao número de nós expandidos. \normalsize
        \caption{Resultados da execução dos vários problemas.}
    \end{table}

    \section{Análise}
    \paragraph{Caracterização do Espaço de Estados}\mbox{} \\
    Tomando um qualquer tabuleiro $b$ e sabendo que uma ação $a$ consiste na remoção de um grupo de cardinalidade $\geq 2$,
    conclui-se que o tabuleiro resultante de aplicar $a$ a $b$ tem estritamente menos peças que o tabuleiro original. Como tal,
    conclui-se que não podem existir ciclos no espaço de estados, pois nunca é possível o tabuleiro voltar a ganhar peças.

    Em acréscimo, uma vez que o número de peças é finito, o comprimento máximo de um caminho no espaço de estados é também finito
    (correspondendo à situação de se removerem todas as peças do tabuleiro, em grupos apenas de 2). Como tal não podem existir
    ramos infinitos no espaço de estados.

    Em último lugar, temos que para cada estado existe um número máximo de sucessores diretos finito, uma vez que cada ação corresponde a
    remover um grupo e o número de grupos é finito. Consequentemente o \textit{branching factor}, $b^*$, é também finito.



    \paragraph{Completude}
    No problema Same Game, qualquer uma das pesquisas estudadas é completa uma vez que o espaço de estados é acíclico e tanto o
    comprimento máximo de um caminho como o \textit{branching factor} são finitos.


    \paragraph{Eficiência}
    Dos 3 métodos de pesquisa abordados, os mais eficientes (em termos de tempo/nós gerados) são claramente as procuras informadas,
    neste caso a Procura Ganaciosa e a Procura A*. Isto deve-se ao facto das procuras informadas usarem heurísticas para direcionar
    a pesquisa em direção ao estado objetivo, reduzindo efetivamente o \textit{branching factor}.
    Por este motivo, a discrepância entre as procuras informadas e não informadas é mais notória quando existe um grande número
    de grupos alternativos que podem ser removidos ($>$ \textit{branching factor}).

    Através da análise de várias instâncias do problema, concluimos ainda, que um dos fatores determinantes para a complexidade
    de uma dada instância é a relação entre o número de cores e o tamanho do tabuleiro, tal como no clássico jogo \textit{minesweeper}
    (neste caso número de bombas vs.\ tamanho do tabuleiro); pois num tabuleiro muito grande com poucas cores a solução é normalmente
    mais simples (uma vez que os grupos são em geral maiores). Em contraste, num tabuleiro pequeno com muitas cores os
    grupos são em geral muito pequenos o que difculta a resolução do problema.


    \paragraph{Heurística}
    Na implementação do nosso problema Same Game usámos a seguinte função heurística:
    \[
        h(n) = \sum_{g\, \in\, groups(n)}
        \begin{cases}
            1, & size(g) \geq 2 \\
            3, & size(g) = 1
        \end{cases}
    \]
    Para cada grupo de cardinalidade $\geq 2$ somamos $1$ ao valor de $h(n)$ uma vez que, no melhor caso, esse grupo irá ser removido
    apenas numa ação; para cada grupo de cardinalidade $1$, somamos $3$ de forma a dirigir a nossa pesquisa de modo a evitar
    grupos de cardinalidade $1$, uma vez que esses grupos são provavelmente mais difíceis de remover que grupos com maiores cardinalidades.
    Experimentalmente constatámos que esta heurística produz ligeiramente menos nós do que não fazendo diferenciação na
    cardinalidade dos grupos.

    Esta heurística é claramente não admissível, uma vez que remover um grupo pode implicar uma alteração radical no número
    de grupos restantes (pois vários grupos separados podem juntar-se num só). No entanto, como este não é um problema de
    optimalidade e sim de satisfação, não é importante que a heurística seja admissível mas sim que conduza o mais rapidamente
    possível ao estado objetivo.

\end{document}
