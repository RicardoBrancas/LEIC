\documentclass[a4paper,twocolumn]{article}
\usepackage[margin=3cm]{geometry}
\usepackage[portuguese]{babel}
\usepackage[utf8]{inputenc}
\usepackage[T1]{fontenc}
\usepackage{extarrows}
\usepackage{tikz}

\usetikzlibrary{automata,positioning,calc}

\title{\LARGE \textbf{Relatório do Projeto 2 de Inteligência Artificial}}
\author{Mihail Brinza \\ \scriptsize 83533 \normalsize \and Ricardo Brancas \\ \scriptsize 83557 \normalsize}

\begin{document}
    \maketitle

    \section{Métodos de Classificação}
    Para classificar as plavras, considerámos as seguintes \textit{features}:
    \begin{enumerate}
        \item Tamanho da palavra;
        \item Número de vogais;
        \item Paridade do número de vogais;
        \item Paridade do número de consoantes;
        \item Número de caracteres ``a'';
    \end{enumerate}
    Testamos, com recurso a \textit{cross validation} os classificadores \textit{k-neighbors}, com $ k = \{1,3,5,7,9\}$
    e \textit{decision tree}, obtendo os resultados indicados na tabela~\ref{tab:cv1}

    \begin{table}[ht]
        \centering
        \begin{tabular}{ l c c }
            & Conjunto 1 & Conjunto 2 \\
         \textit{1-neighbor}  & $92.54\%$ & $86.75\%$ \\
         \textit{3-neighbors} & $85.79\%$ & $85.79\%$ \\
         \textit{5-neighbors} & $86.76\%$ & $85.55\%$ \\
         \textit{7-neighbors} & $85.79\%$ & $83.87\%$ \\
         \textit{9-neighbors} & $85.79\%$ & $85.32\%$ \\
         \textit{Decision Tree} & $100\%$ & $98.80\%$
        \end{tabular}
        \caption{Resultados da validação cruzada para o problema 1.}
        \label{tab:cv1}
    \end{table}

    \section{Métodos de Regressão}

    \section{Aprendizagem por Reforço}
    \subsection{Trajetórias Aprendidas}
    \subsubsection{Exemplo 1}
    $ 5_0 \xlongrightarrow{0} 6_1 \xlongrightarrow{0} 6_1 \xlongrightarrow{0} 6_1 \xlongrightarrow{0} 6_{(1)} $

    \subsubsection{Exemplo 2}
    $ 5_0 \xlongrightarrow{0} 6_1 \xlongrightarrow{0} 1_0 \xlongrightarrow{1} 0_1 \xlongrightarrow{1} 0_{(1)} $


    \subsection{Modelo do Mundo}
    \subsubsection{Exemplo 1}
    Neste primeiro exemplo o ambiente consiste numa série de quadrículas sequênciais, tal como demonstrado na
    figura~\ref{fig:amb1}, em que a ação $1$ corresponde a dar um passo para a esquerda e a ação $0$ corresponde a
    dar um passo para a direita. Tentar andar para fora dos limites não tem qualquer efeito.
    Para além disto, no estado 5 a ação 0 não é determinística; ao realizar esta ação, o agente pode ir ter ao estado
    6 ou permanecer no estado 5.

    \begin{figure}[ht]
        \centering
        \begin{tikzpicture}[shorten >=1pt,node distance=1.8cm,on grid,auto]
            \node[state,accepting] (q_0)   {0};
            \node[state] (q_1) [right=of q_0] {1};
            \node[state] (q_2) [right=of q_1]   {2};
            \node[state] (q_3) [above right=of q_2]   {3};
            \node[state] (q_4) [above left=of q_3]   {4};
            \node[state] (q_5) [left =of q_4]   {5};
            \node[state,accepting] (q_6) [left =of q_5]   {6};
            \path[->]
            (q_0) edge [bend left=20] node {0} (q_1)
                  edge [loop below] node {1} ()
            (q_1) edge [bend left=20] node {0} (q_2)
                  edge [bend left=20] node {1} (q_0)
            (q_2) edge [bend left=20] node {0} (q_3)
                  edge [bend left=20] node {1} (q_1)
            (q_3) edge [bend left=20] node {0} (q_4)
                  edge [bend left=20] node {1} (q_2)
            (q_4) edge [bend left=20] node {0} (q_5)
                  edge [bend left=20] node {1} (q_3)
            (q_5) edge [bend left=20] node {0} (q_6)
                  edge [bend left=20] node {1} (q_4)
                  edge [loop below] node {0} (q_5)
            (q_6) edge [loop below] node {0} (q_6)
                  edge [bend left=20] node {1} (q_5);
        \end{tikzpicture}
        \caption{Ambiente 1. Os nós com duplo contorno são os nós de recompensa.} \label{fig:amb1}
    \end{figure}

    Por fim determinámos que o agente é recompensado ($r = 1$) sempre que o estado
    inicial (antes da ação) é um estado limite. (i.e.~0 ou 6), não sendo recompensado
    nas outras situações ($r = 0$).

    Como tal, quando começamos no estado 5 o melhor curso de ação é andar rapidamente
    para o estado limite mais próximo, o 6, e depois mantermo-nos lá.

    \subsubsection{Exemplo 2}
    No segundo exemplo o ambiente é muito semelhante ao primeiro com exceção do estado 6. Agora quando
    tentamos andar para a direita (ação 0) nesse estado voltamos para o estado 1, tal como representado na figura~\ref{fig:amb2}.
    A função de recompensa mantem-se também inalterada, sendo 1 quando o estado inicial é o zero ou o seis, e
    0 caso contrário.

    \begin{figure}[ht]
        \centering
        \begin{tikzpicture}[shorten >=1pt,node distance=1.8cm,on grid,auto]
            \node[state,accepting] (q_0)                {0};
            \node[state]           (q_1) [right=of q_0] {1};
            \node[state]           (q_2) [right=of q_1] {2};
            \node[state]           (q_3) [right=of q_2] {3};
            \node[state]           (q_4) [above=of q_3] {4};
            \node[state]           (q_5) [left=of q_4]  {5};
            \node[state,accepting] (q_6) [left=of q_5]  {6};
            \path[->]
            (q_0) edge [bend left=20] node {0} (q_1)
                  edge [loop above] node {1} ()
            (q_1) edge [bend left=20] node {0} (q_2)
                  edge [bend left=20] node {1} (q_0)
            (q_2) edge [bend left=20] node {0} (q_3)
                  edge [bend left=20] node {1} (q_1)
            (q_3) edge [bend left=20] node {0} (q_4)
                  edge [bend left=20] node {1} (q_2)
            (q_4) edge [bend left=20] node {0} (q_5)
                  edge [bend left=20] node {1} (q_3)
            (q_5) edge [bend left=20] node {0} (q_6)
                  edge [bend left=20] node {1} (q_4)
                  edge [loop above] node {0} (q_5)
            (q_6) edge             node {0} (q_1)
                  edge [bend left=20] node {1} (q_5);
        \end{tikzpicture}
        \caption{Ambiente 2. Os nós com duplo contorno são os nós de recompensa.} \label{fig:amb2}
    \end{figure}

    Como tal [quando começamos no estado 5] já não é possível usar a estratégia anterior de ficar parado no estado 6.
    Assim, o melhor a fazer nesta situação é tentermo-nos dirigir para o estado 0, de modo a maximizar a recompensa a
    longo prazo, que é exatamente o que obtemos com o algoritmo Q-learning.

\end{document}