\documentclass[a4paper]{article}
\usepackage[margin=2.5cm]{geometry}
\usepackage[portuguese]{babel}
\usepackage[utf8]{inputenc}
\usepackage[T1]{fontenc}
\usepackage{multirow}
\usepackage{siunitx}

\title{\LARGE \textbf{Relatório Inteligência Artificial}}
\author{Mihail Brinza \\ \scriptsize 83533 \normalsize \and Ricardo Brancas \\ \scriptsize 83557 \normalsize}

\begin{document}
    \maketitle

    \section{Resultados Exeprimentais}

    \subsection{Breve descrição dos problemas utilizados}

    \subparagraph{Problema 1}
    [4x5], 2 cores, sem solução

    \subparagraph{Problema 2}
    [4x5], 3 cores

    \subparagraph{Problema 3}
    [10x4], 5 cores, sem solução

    \subparagraph{Problema 4}
    [10x4], 3 cores

    \subparagraph{Problema 5}
    [10x4], 5 cores

    \subsection{Resultados}

    \begin{table}[h!]
        \begin{tabular}{ c|*{3}{c}|*{3}{c}|*{3}{c} }
            & \multicolumn{3}{ c| }{DFS} & \multicolumn{3}{ c| }{Greedy} & \multicolumn{3}{ c }{A*} \\ \cline{2-10}
            Problema & t & g & e & t & g & e & t & g & e \\
            \hline
            1 & 0.0001 & 0 & 1 & 0.0002 & 0 & 1 & 0.0002 & 0 & 1     \\
            2 & 0.0004 & 7 & 4 & 0.0006 & 6 & 3 & 0.0006 & 7 & 4     \\
            3 & 5.8729 & 74701 & 74702 & 12.091 & 74701 & 74702 & 12.030 & 74701 & 74702 \\
            4 & 0.0052 & 85 & 54 & 0.0105 & 59 & 42 & 0.0058 & 43 & 24    \\
            5 & 221.30 & 3123363 & 3123308 & 0.0357 & 237 & 169 & 0.0145 & 91 & 16    \\
        \end{tabular}
        \caption{Resultados da execução dos vários problemas. \small \textbf{t} refere-se ao tempo demorado; \textbf{g} ao número de nós gerados e \textbf{e} ao número de nós expandidos.}
    \end{table}

    \section{Análise}
    \paragraph{Caracterização do Espaço de Estados}\mbox{} \\
    Tomando um qualquer tabuleiro $b$ e sabendo que uma ação $a$ consiste na remoção de um grupo de cardinalidade $\geq 2$,
    conclui-se que o tabuleiro resultante de aplicar $a$ a $b$ tem estritamente menos peças que o tabuleiro original. Como tal,
    conclui-se que não podem existir ciclos no espaço de estados, pois nunca é possível o tabuleiro voltar a ganhar peças.

    Em acréscimo, uma vez que o número de peças é finito, o comprimento máximo de um caminho no espaço de estados é também finito
    (correspondendo à situação de se removerem todas as peças do tabuleiro, em grupos apenas de 2). Como tal não podem existir
    ramos infinitos no espaço de estados.

    Em último lugar, temos que para cada estado existe um número máximo de sucessores diretos finito, uma vez que cada ação corresponde a
    remover um grupo e o número de grupos é finito. Consequentemente o \textit{branching factor}, $b^*$, é também finito.



    \paragraph{Completude}
    No problema Same Game, qualquer uma das pesquisas estudadas é completa uma vez que o espaço de estados é acíclico e o
    comprimento máximo de um caminho e o \textit{branching factor} são finitos.


\end{document}
