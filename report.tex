\documentclass[a4paper,twocolumn]{article}
\usepackage[margin=2.5cm]{geometry}
\usepackage[portuguese]{babel}
\usepackage[utf8]{inputenc}
\usepackage[T1]{fontenc}
\usepackage{extarrows}
\usepackage{pgfplots}
\usepackage{xcolor}
\usepackage{tikz}

\usetikzlibrary{automata,positioning,calc}
\usepgfplotslibrary{groupplots}

\setlength{\columnsep}{0.75cm}
\pgfplotsset{compat=1.13}

\pgfplotsset{every axis/.append style={
        scale only axis,
        width=0.4\textwidth,
        height=0.3\textwidth
    },every axis legend/.append style={
        at={(0.5,-.18)},
        anchor=north
    }}

\definecolor{lightblue}{RGB}{205, 236, 247}

\pgfplotscreateplotcyclelist{mycolorlist}{%
brown!60!black\\%
black\\%
blue\\%
red\\%
green!60!black\\%
}

\title{\LARGE \textbf{Relatório do Projeto 2 de Inteligência Artificial}}
\author{Mihail Brinza \\ \scriptsize 83533 \normalsize \and Ricardo Brancas \\ \scriptsize 83557 \normalsize}

\begin{document}
    \maketitle

    \section{Métodos de Classificação}
    Para classificar as plavras escolhemos de entre outras, com recurso a \textit{cross validation}, as seguintes \textit{features}:
    \begin{enumerate}
        \item Paridade do número de caracteres;
        \item Paridade do número de caracteres ``z'';
        \item Paridade do número de vogais;
        \item Paridade do número de consoantes;
        \item Número de caracteres ``a''.
    \end{enumerate}
    Para escolher o classificador, usámos novamente \textit{cross validation} com os classificadores \textit{k-neighbors},
    com $ k = \{1,3,5,7,9\}$ e \textit{decision tree}, obtendo os resultados indicados na tabela~\ref{tab:cv1}.

    \begin{table}[ht]
        \centering
        \begin{tabular}{ l c c }
            & Conjunto 1 & Conjunto 2 \\
         \textit{1-neighbor}    & $1.0$ & $1.0$ \\
         \textit{3-neighbors}   & $1.0$ & $1.0$ \\
         \textit{5-neighbors}   & $1.0$ & $1.0$ \\
         \textit{7-neighbors}   & $1.0$ & $1.0$ \\
         \textit{9-neighbors}   & $1.0$ & $1.0$ \\
         \textit{Decision Tree} & $1.0$ & $1.0$
        \end{tabular}
        \caption{\textit{$F_1$ scores} da validação cruzada para o problema 1.}
        \label{tab:cv1}
    \end{table}

    Concluímos portanto que, para estas \textit{features}, qualquer um dos classificadores testados escolhe sempre bem dentro do
    conjunto de treino. Mais tarde verificámos que a escolha também é sempre acertada no conjunto de testes.

    Decidimos utilizar o \textit{Decision Tree Classifier} porque escolhe corretamente com maior probabilidade mesmo quando
    utilizamos \textit{features} piores.

    \section{Métodos de Regressão}
    Para escolher o método de regressão mais apropriado usamos \textit{cross validation} pontuado pelo erro quadrático
    médio, obtendo os resultados dispostos na tabela~\ref{tab:cv2}.

    \begin{table}[ht]
        \centering
        \begin{tabular}{ l c c }
            & $g_1()$ & $g_2()$ \\
         \textit{Linear Regression}                                      & $0.94$ & $816$ \\
         \textit{KRR\footnotemark[1]} ($\gamma = 0.05$, $\alpha = 0.1$)   & $0.67$ & $1235$ \\
         \textit{KRR\footnotemark[1]} ($\gamma = 0.05$, $\alpha = 0.01$)  & $0.34$ & \colorbox{lightblue}{$707$} \\
         \textit{KRR\footnotemark[1]} ($\gamma = 0.05$, $\alpha = 0.001$) & \colorbox{lightblue}{$0.14$} & \colorbox{lightblue}{$428$} \\
         \textit{KRR\footnotemark[1]} ($\gamma = 0.1$,  $\alpha = 0.1$)   & \colorbox{lightblue}{$0.23$} & $1265$ \\
         \textit{KRR\footnotemark[1]} ($\gamma = 0.1$,  $\alpha = 0.01$)  & \colorbox{lightblue}{$0.10$} & $811$ \\
         \textit{KRR\footnotemark[1]} ($\gamma = 0.1$,  $\alpha = 0.001$) & \colorbox{lightblue}{$0.10$} & \colorbox{lightblue}{$548$} \\
         \textit{KRR\footnotemark[1]} ($\gamma = 0.2$,  $\alpha = 0.1$)   & $0.40$ & $1445$ \\
         \textit{KRR\footnotemark[1]} ($\gamma = 0.2$,  $\alpha = 0.01$)  & \colorbox{lightblue}{$0.24$} & $1104$ \\
         \textit{KRR\footnotemark[1]} ($\gamma = 0.2$,  $\alpha = 0.001$) & $0.66$ & \colorbox{lightblue}{$799$} \\
         \textit{KRR\footnotemark[2]} (degree $= 2$)                      & $2.27$ & $3589$ \\
         \textit{KRR\footnotemark[2]} (degree $= 3$)                      & $7.75$ & \colorbox{lightblue}{$0.38$} \\
         \textit{KRR\footnotemark[2]} (degree $= 4$)                      & $0.93$ & \colorbox{lightblue}{$1.25$} \\
         \textit{KRR\footnotemark[2]} (degree $= 5$)                      & $21.72$ & \colorbox{lightblue}{$4.39$} \\
         \textit{Decision Tree}                                          & $0.73$ & $1290$ \\
        \end{tabular}
        \caption{\textit{MSE scores} da validação cruzada para o problema 2.
        Os erros dentro do \textit{threshold} definido estão marcados a azul.}
        \label{tab:cv2}
    \end{table}
    \footnotetext[1]{\textit{Kernel Ridge Regression} com \textit{radial basis function kernel}.}
    \footnotetext[2]{\textit{Kernel Ridge Regression} com \textit{polynomial kernel}.}

    Das duas parametrizações testadas que apresentam resulados aceitáveis para ambas as funções decidimos utilizar
    a \textit{Kernel Ridge Regression} com \textit{kernels} do tipo \textit{radial basis function} e parâmetros
    \{$\gamma = 0.1$,  $\alpha = 0.001$\}. Apresentamos na figura~\ref{fig:reg} alguns dos resultados obtidos,
    em particular para a regressão escolhida.

    \begin{figure}[ht]
        \centering
        \begin{tikzpicture}[trim axis left, trim axis right]
            \begin{groupplot}[group style={group size= 1 by 2,vertical sep=1.5cm},cycle list name=mycolorlist]
                \nextgroupplot[title=$g_1()$]
                    \addplot+ [only marks, mark size=0.6pt] table [x = xp, y = yp, col sep=comma] {p2_0.csv};
                    \addplot+ [only marks, mark size=3pt, mark=+] table [x = x, y = y, col sep=comma] {p2_0.csv};
                    \addplot+ [mark=none,thick] table [x = xp, y = ypred1, col sep=comma] {p2_0.csv};
                    \addplot+ [mark=none,thick] table [x = xp, y = ypred2, col sep=comma] {p2_0.csv};
                    \addplot+ [mark=none,thick] table [x = xp, y = ypred3, col sep=comma] {p2_0.csv};
                \nextgroupplot[title=$g_2()$]
                    \addplot+ [only marks, mark size=0.6pt] table [x = xp, y = yp, col sep=comma] {p2_1.csv};
                    \addlegendentry{Dados de teste}
                    \addplot+ [only marks, mark size=3pt, mark=+] table [x = x, y = y, col sep=comma] {p2_1.csv};
                    \addlegendentry{Dados de treino}
                    \addplot+ [mark=none,thick] table [x = xp, y = ypred1, col sep=comma] {p2_1.csv};
                    \addlegendentry{KRR (rbf, $\gamma = 0.1$,  $\alpha = 0.001$)}
                    \addplot+ [mark=none,thick] table [x = xp, y = ypred2, col sep=comma] {p2_1.csv};
                    \addlegendentry{KRR (poly, degree $= 3$)}
                    \addplot+ [mark=none,thick] table [x = xp, y = ypred3, col sep=comma] {p2_1.csv};
                    \addlegendentry{Linear Regression}
            \end{groupplot}
        \end{tikzpicture}
        \caption{Resultados obtidos para as regressões.}
        \label{fig:reg}
    \end{figure}

    \section{Aprendizagem por Reforço}
    \subsection{Trajetórias Aprendidas}
    \subsubsection{Ambiente 1}
    $ 5_0 \xlongrightarrow{0} 6_1 \xlongrightarrow{0} 6_1 \xlongrightarrow{0} 6_1 \xlongrightarrow{0} 6_{(1)} $

    \subsubsection{Ambiente 2}
    $ 5_0 \xlongrightarrow{0} 6_1 \xlongrightarrow{0} 1_0 \xlongrightarrow{1} 0_1 \xlongrightarrow{1} 0_{(1)} $


    \subsection{Modelo do Mundo}
    \subsubsection{Ambiente 1}
    Apresentamos na figura~\ref{fig:amb1} um esquema do ambiente 1. Os movimentos são, maioritariamente, sequênciais
    com exceção da ação 0 no estado 5 que é não-determinística. Tentar andar para os estados anterioes/seguintes
    a partir dos estados 0/6, respetivamente, não tem qualquer efeito.

    \begin{figure}[ht]
        \centering
        \begin{tikzpicture}[shorten >=1pt,node distance=1.8cm,on grid,auto]
            \node[state,accepting] (q_0)   {0};
            \node[state] (q_1) [right=of q_0] {1};
            \node[state] (q_2) [right=of q_1]   {2};
            \node[state] (q_3) [above right=of q_2]   {3};
            \node[state] (q_4) [above left=of q_3]   {4};
            \node[state] (q_5) [left =of q_4]   {5};
            \node[state,accepting] (q_6) [left =of q_5]   {6};
            \path[->]
            (q_0) edge [bend left=20] node {0} (q_1)
                  edge [loop below] node {1} ()
            (q_1) edge [bend left=20] node {0} (q_2)
                  edge [bend left=20] node {1} (q_0)
            (q_2) edge [bend left=20] node {0} (q_3)
                  edge [bend left=20] node {1} (q_1)
            (q_3) edge [bend left=20] node {0} (q_4)
                  edge [bend left=20] node {1} (q_2)
            (q_4) edge [bend left=20] node {0} (q_5)
                  edge [bend left=20] node {1} (q_3)
            (q_5) edge [bend left=20] node {0} (q_6)
                  edge [bend left=20] node {1} (q_4)
                  edge [loop below] node {0} (q_5)
            (q_6) edge [loop below] node {0} (q_6)
                  edge [bend left=20] node {1} (q_5);
        \end{tikzpicture}
        \caption{Ambiente 1. Os nós com duplo contorno são os nós de recompensa.} \label{fig:amb1}
    \end{figure}

    A função de recompensa é a seguinte:
    \[
        r(s) = \left\{
        \begin{array}{ll}
              1, & s \in \{0,6\} \\
              0, & s \notin \{0,6\}
        \end{array}
        \right.
    \]

    Assim, a melhor ação a tomar é sempre dirigirmo-nos o mais rapidamente possível para um dos estados de recompensa,
    tendo eventualmente cuidado com o não-determinismo do estado 5.
    Começando no estado 5, a melhor ação é sem duvida movermo-nos para o estado 6 e depois mantermo-nos lá.

    \subsubsection{Ambiente 2}
    O ambiente 2 é muito semelhante ao primeiro com exceção do estado 6. Agora quando
    tentamos tomar a ação 0 nesse estado voltamos para o estado 1, tal como representado na figura~\ref{fig:amb2}.

    \begin{figure}[ht]
%        \centering
        \begin{center}
            \begin{tikzpicture}[shorten >=1pt,node distance=1.8cm,on grid,auto]
                \node[state,accepting] (q_0)                {0};
                \node[state]           (q_1) [right=of q_0] {1};
                \node[state]           (q_2) [right=of q_1] {2};
                \node[state]           (q_3) [right=of q_2] {3};
                \node[state]           (q_4) [above=of q_3] {4};
                \node[state]           (q_5) [left=of q_4]  {5};
                \node[state,accepting] (q_6) [left=of q_5]  {6};
                \path[->]
                (q_0) edge [bend left=20] node {0} (q_1)
                      edge [loop above] node {1} ()
                (q_1) edge [bend left=20] node {0} (q_2)
                      edge [bend left=20] node {1} (q_0)
                (q_2) edge [bend left=20] node {0} (q_3)
                      edge [bend left=20] node {1} (q_1)
                (q_3) edge [bend left=20] node {0} (q_4)
                      edge [bend left=20] node {1} (q_2)
                (q_4) edge [bend left=20] node {0} (q_5)
                      edge [bend left=20] node {1} (q_3)
                (q_5) edge [bend left=20] node {0} (q_6)
                      edge [bend left=20] node {1} (q_4)
                      edge [loop above] node {0} (q_5)
                (q_6) edge             node {0} (q_1)
                      edge [bend left=20] node {1} (q_5);
            \end{tikzpicture}
        \end{center}
        \caption{Ambiente 2. Os nós com duplo contorno são os nós de recompensa.} \label{fig:amb2}
    \end{figure}

    A função de recompensa mantem-se também inalterada:
     \[
        r(s) = \left\{
        \begin{array}{ll}
              1, & s \in \{0,6\} \\
              0, & s \notin \{0,6\}
        \end{array}
        \right.
    \]

    Como tal [quando começamos no estado 5] já não é possível usar a estratégia anterior de ficar parado no estado 6.
    Assim, o melhor a fazer nesta situação é tentermo-nos dirigir para o estado 0, de modo a maximizar a recompensa a
    longo prazo, que é exatamente o que obtemos com o algoritmo Q-learning.

\end{document}