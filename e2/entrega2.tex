\documentclass[a4paper]{article}
\usepackage[margin=3cm]{geometry}
\usepackage[portuguese]{babel}
\usepackage[utf8]{inputenc}
\usepackage[T1]{fontenc}
\usepackage{graphicx}
\usepackage{amsmath}
\usepackage{courier}
\usepackage{listings}

\newcommand{\var}[1]{\textit{#1}}
\newcommand{\rel}[1]{\textbf{\var{#1}}}
\newcommand{\PK}[1]{\underline{\smash{\var{#1}}}}
\newcommand{\restrict}{\hspace*{2em}}
\newcommand{\FK}[2]{\restrict \var{#1}: \textup{FK}(\var{#2})}
\newcommand{\FKfull}[3]{\restrict \var{#1}: \textup{FK}(\var{#2}.\var{#3})}
\newcommand{\RI}{\restrict RI: }
\newcommand{\outRI}{RI: }

\newenvironment{relationalmodel}{\setlength{\parindent}{0cm}}

\lstset{basicstyle=\footnotesize\ttfamily,breaklines=true}

\begin{document}
    \begin{titlepage}
        \centering
        \includegraphics[width=0.5\textwidth]{IST_A_CMYK_POS.pdf}\par
        {\huge\bfseries Projeto de Bases de Dados, Parte 2\par}
        \vspace{2cm}
        {
        \Large
        \begin{tabular}{llll}
            83533 & Mihail Brinza & 24\% & 4h \\
            38557 & Ricardo Brancas & 64\% & 11h \\
            83883 & David Nunes & 12\% & 2h
        \end{tabular}
        }
        \vfill
        \large
        Grupo 37, turno BD2251795L10 \par
        Professor Miguel Amaral

        \vspace{3cm}

        {\normalsize \today\par}
    \end{titlepage}

    \section{Modelo Relacional}
    \begin{relationalmodel}

        \rel{Produto}(\PK{ean}, \var{design})\\
        \RI um produto só é válido se participar na relação \rel{fornece\_prim}\\
        \RI um produto só é válido se participar na relação \rel{fornece\_sec}\\
        \RI um produto só é válido se participar na relação \rel{tem}\\

        \rel{Corredor}(\PK{nro}, \var{largura})\\

        \rel{Prateleira}(\PK{nro, lado, altura})\\
        \FK{nro}{Corredor}\\

        \rel{planograma}(\PK{ean, nro, lado, altura}, \var{faces}, \var{unidades}, \var{loc})\\
        \FK{ean}{Produto}\\
        \FK{nro, lado, altura}{Prateleira}\\

        \rel{Fornecedor}(\PK{nif}, \var{nome})\\

        \rel{fornece\_prim}(\PK{ean}, \var{nif}, \var{data})\\
        \FK{ean}{Produto}\\
        \FK{nif}{Fornecedor}\\

        \rel{fornece\_sec}(\PK{ean, nif})\\
        \FK{ean}{Produto}\\
        \FK{nif}{Fornecedor}\\

        \outRI O mesmo par (\var{ean}, \var{nif}) não pode existir nas relações \rel{fornece\_prim} e \rel{fornece\_sec} ao mesmo tempo\\

        \rel{Categoria}(\PK{nome})\\
        \RI \var{nome} tem que existir em \rel{Categoria Simples} ou (exclusivo) \rel{Super Categoria}\\

        \rel{Categoria Simples}(\PK{nome})\\
        \FK{nome}{Categoria}\\

        \rel{Super Categoria}(\PK{nome})\\
        \FK{nome}{Categoria}\\
        \RI uma super categoria só é válida se participar na relação \rel{constituida}\\

        \rel{constituida}(\PK{sub\_nome, super\_nome})\\
        \FKfull{sub\_nome}{Categoria}{nome}\\
        \FKfull{super\_nome}{Categoria}{nome}\\
        \RI não podem existir ciclos (diretos ou indiretos) nesta relação\\ %TODO

        \rel{tem}(\PK{ean}, \var{nome})\\
        \FK{ean}{Produto}\\
        \FK{nome}{Categoria}\\

        \rel{Evento Reposição}(\PK{operador, instante})\\
        \RI um evento de reposição só é válido se participar na relação \rel{reposição}\\

        \rel{reposição}(\PK{operador, instante, ean, nro, lado, altura}, \var{unidades})\\
        \FK{operador, instante}{Evento Reposição}\\
        \FK{ean, nro, lado, altura}{planograma}\\
        \RI o momento de reposição tem de ser anterior ao momento atual\\

        \outRI Para cada elemento de \rel{reposição} as \var{unidades} têm de ser inferiores às \var{unidades}
        do elemento de \rel{palonograma} associado.\\
    \end{relationalmodel}

    \section{Algebra Relacional}

    \begin{enumerate}
        \setlength\itemsep{1em}

        \item $\Pi_{ean, design}(Produto \bowtie \sigma_{nome='Fruta'}(tem) \bowtie \\ \sigma_{sum\_unidades > 10}(_{ean}G_{\textsc{Sum}(unidades) \mapsto sum\_unidades}(\sigma_{data > '10/1/2017'}(reposicao)))) $
        \item $ (\Pi_{nif}(\sigma_{ean=x}(Produto) \bowtie fornece\_prim) \cup \Pi_{nif}(\sigma_{ean=x}(Produto) \bowtie fornece\_sec)) \bowtie Fornecedor $
        \item $ G_{\textsc{Count()}}(\sigma_{super\_nome='Congelados'}(constituida)) $
        \item $ cats\_por\_nif \gets ~_{nif}G_{\textsc{Count}() \mapsto count\_nome}(\Pi_{nif, nome}((\Pi_{ean, nif}(fornce\_prim) \cup \\ fornece\_sec) \bowtie tem)) \\
        \sigma_{max = count\_nome}(G_{\textsc{Max}(count\_nome) \mapsto max}(cats\_por\_nif) \times cats\_por\_nif) \bowtie Fornecedor $
        \item $  (\Pi_{nif, nome}(fornece\_prim \bowtie tem) \div Categoria Simples) \bowtie Fornecedor $
        \item $ \sigma_{nro, largura}((planograma \bowtie fornece\_prim) \div (\Pi_{nif}(fornece\_prim) - \Pi_{nif}(fornece\_sec))) $
    \end{enumerate}

    \section{SQL}
    \lstinputlisting[language=SQL]{entrega2.sql}
\end{document}
