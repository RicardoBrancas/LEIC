\documentclass[a4paper]{article}
\usepackage[margin=3cm]{geometry}
\usepackage[portuguese]{babel}
\usepackage[utf8]{inputenc}
\usepackage[T1]{fontenc}
\usepackage{graphicx}
\usepackage{amsmath}
\usepackage{courier}
\usepackage{listings}
\usepackage{tikz}
\usepackage{tikz-qtree}
\usepackage{enumitem}

\lstset{language=SQL}
\lstset{basicstyle=\footnotesize\ttfamily,breaklines=true,morekeywords={IF,REFERENCES,TYPE,ENUM,REPLACE,FUNCTION,RETURNS,VOID,DECLARE,BEGIN,DEFERRED,FOR,LOOP,LANGUAGE,RAISE,SETOF,WHILE,RETURN,QUERY,TEMP}}

\begin{document}

    \begin{titlepage}
        \centering
        \includegraphics[width=0.5\textwidth]{IST_A_CMYK_POS.pdf}\par
        {\huge\bfseries Projeto de Bases de Dados, Parte 3\par}
        \vspace{2cm}
        {
        \Large
        \begin{tabular}{llll}
            83533 & Mihail Brinza & .\% & h \\
            38557 & Ricardo Brancas & .\% & h \\
            83883 & David Nunes & .\% & h
        \end{tabular}
        }
        \vfill
        \large
        Grupo 37, turno BD2251795L10 \par
        Professor Miguel Amaral

        \vspace{3cm}

        {\normalsize \today\par}
    \end{titlepage}

    \section{Criação da Base de Dados}
    \lstinputlisting{schema.sql}

    \section{SQL}
    \lstinputlisting{queries.sql}

    \section{Desenvolvimento da Aplicação}
    A nossa aplicação em PHP segue o modelo típico cliente-servidor, em que a parte cliente se liga e faz pedidos ao SGBD
    via a interface PDO (PHP Data Objects).
    A página inicial, \texttt{index.html}, contem hiperligações para 5 outras páginas, cada uma correspondente
    a uma das alíneas pedidas; tal como demonstrado no esquema seguinte:

    \begin{center}
        \Tree[ .index.html [ .a.php [ .do\_a.php ] ]
        [.b.php [ .do\_b.php ] ]
        [ .c.php [ .do\_c.php ] ]
        [ .d.php [ .do\_d.php ] ]
        [ .e.php [ .do\_e.php ] ] ]
    \end{center}


    Para fazer os pedidos às respetivas páginas de execução (\texttt{do\_*.php}) utilizámos o verbo \texttt{GET} no caso das alíneas
    \textbf{c} e \textbf{e}, por se tratar de um pedido sem ``efeitos secundários'' e \texttt{POST} nos restantes, por realizarem
    modificaçoes à base de dados.

    De um modo geral, em cada uma das páginas de input, existem formulários que o utilizador preenche e que, ao serem submetidos,
    enviam um pedido para a respetiva página de ação. Nessa página verificamos sumariamente o input do utilizador
    e executamos os \textit{statements} correspondentes, utilizando transações quando se trata de uma ação \textit{multi-statement} e
    fazendo \textit{rollback} quando necessário.

    No fim, em caso de sucesso, é mostrado ao utilizador qual o output, caso exista, e as \textit{queries} que foram executadas.
    Se acontecer um erro, o mesmo é mostrado.

    Descrevemos em seguida os detalhes de funcionamento para as alínas relevantes:

    \begin{enumerate}[label=\textbf{\alph*)}]
        \item Inserir e remover categorias e sub-categorias;

        Na página de input (\texttt{a.php}) existem 3 opções diponíveis para o utilizador: inserir uma categoria (simples),
        inserir uma subcategoria ou remover uma categoria.

        Consoante o formulário em que o utilizador faça \textit{submit}, diferentes parâmetros vão ser passados no \texttt{POST}
        o que vai corresponder a ações diferentes realizadas no \texttt{do\_a.php}.

        \item Inserir e remover um novo produto e os seus respetivos fornecedores (primários ou secundários) garantindo que esta operação seja atómica;

        São, novamente, dadas duas opções ao utilizador: inserir ou remover o produto. Tal como na alínea anteriror,
        a form submetida define qual a ação a ser realizada.

%        \item Listar eventos de reposição de um dado produto, incluindo o número de unidades respostas;
%        \item Alterar a designação de um produto;
%        \item Listar todas as subcategorias de uma super-categoria, a todos os níveis de profundidade.
    \end{enumerate}


\end{document}