\documentclass[a4paper]{article}
\usepackage[margin=3cm]{geometry}
\usepackage[portuguese]{babel}
\usepackage[utf8]{inputenc}
\usepackage[T1]{fontenc}
\usepackage{graphicx}
\usepackage{amsmath}
\usepackage{courier}
\usepackage{listings}
\usepackage{tikz}
\usepackage{tikz-qtree}

\lstset{language=SQL}
\lstset{basicstyle=\footnotesize\ttfamily,breaklines=true,morekeywords={IF,REFERENCES,TYPE,ENUM,REPLACE,FUNCTION,RETURNS,VOID,DECLARE,BEGIN,DEFERRED,FOR,LOOP,LANGUAGE,RAISE,SETOF,WHILE,RETURN,QUERY,TEMP}}

\begin{document}

    \begin{titlepage}
        \centering
        \includegraphics[width=0.5\textwidth]{IST_A_CMYK_POS.pdf}\par
        {\huge\bfseries Projeto de Bases de Dados, Parte 3\par}
        \vspace{2cm}
        {
        \Large
        \begin{tabular}{llll}
            83533 & Mihail Brinza & .\% & h \\
            38557 & Ricardo Brancas & .\% & h \\
            83883 & David Nunes & .\% & h
        \end{tabular}
        }
        \vfill
        \large
        Grupo 37, turno BD2251795L10 \par
        Professor Miguel Amaral

        \vspace{3cm}

        {\normalsize \today\par}
    \end{titlepage}

    \section{Criação da Base de Dados}
    \lstinputlisting{schema.sql}

    \section{SQL}
    \lstinputlisting{queries.sql}

    \section{Desenvolvimento da Aplicação}
    Na nossa aplicação existe uma página \texttt{index.html} que contem hiperligações para 5 páginas, cada uma correspondente
    a uma das alíneas pedidas; tal como demonstrado no esquema seguinte:

    \begin{center}
        \Tree[ .index.html [ .a.php [ .do\_a.php ] ] %
        [.b.php [ .do\_b.php ] ] %
        [ .c.php [ .do\_c.php ] ]%
        [ .d.php [ .do\_d.php ] ]%
        [ .e.php [ .do\_e.php ] ] ]
    \end{center}


    Para fazer os pedidos às respetivas páginas de execução (\texttt{do\_*.php}) utilizámos o verbo \texttt{GET} no caso das alíneas
    \textbf{c} e \textbf{e}, por se tratar de um pedido sem "efeitos secundários" e \texttt{POST} nos restantes, por realizarem
    modificaçoes à base de dados.


\end{document}