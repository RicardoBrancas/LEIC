\documentclass[a4paper]{article}
\usepackage[margin=1.35in]{geometry}
\usepackage[portuguese]{babel}
\usepackage[utf8]{inputenc}
\usepackage[T1]{fontenc}
\usepackage{pgfplots}
\usepackage{pgfplotstable}
\usepackage{mathtools}
\usepackage{clrscode3e}
\pgfplotsset{compat=1.13}

\title{\LARGE \textbf{Relatório ASA}}
\author{Mihail Brinza \\ \scriptsize 83533 \normalsize \and Ricardo Brancas \\ \scriptsize 83557 \normalsize}

\begin{document}
	\maketitle
	\section{Introdução}
	Este projeto tem como objetivo criar um sistema que permita ao utilizador conhecer
	a forma mais barata de ligar um conjento de cidades com base numa rede de possíveis
	infra-estruturas.
	O utilizador dá como input ao sistema o número de cidades, o número de possiveis aeroportos,
	o custo de cada um deles, o número de estradas e o custo de cada uma.
	O software indica ainda caso não seja possível interligar todas as cidades.
	No caso de existirem duas formas de interligar as cidades com o mesmo custo,
	é indicada aquela que utiliza menos aeroportos.

	\section{Descrição da Solução}
	Para resolver o problema de encontrar a forma mais barata de interligar $c$
	cidades, considerando $a$ possíveis aeroportos e $e$ possiveis estradas,
	consideramos um grafo não dirigido $G = (V, E, w)$ em que:
	\begin{enumerate}
		\item $|V| = c + 1$
		\item $|E| = a + e$
		\item $w$ é a função de pesos que faz corresponder a cada arco o custo
		 de construir a estrada/aeroporto.
	\end{enumerate}
	Neste grafo uma estrada entre as cidades $i$ e $j$ corresponde a um arco
	$(v_i, v_j)$; e existir um aeroporto na cidade $i$ corresponde a existir
	um arco $(v_0, v_i)$, onde $v_0$ é um vértice especial que representa as
	ligações entre todos os aeroportos (como se fosse o céu).

	Considerando este grafo, encontrar a solução do problema passa por descobrir
	a MST do grafo, sendo apenas necessário tomar especial atenção ao requerimento
	de que deve ser escolhida a MST que utiliza menos aeroportos (na realidade é
	necessário descobrir duas MST, uma considerando só as estradas e outra considerando
	todos os arcos).

	\section{Análise Teórica}
	Para encontrar a(s) MST utilizámos o Algoritmo de Kruskal. Consideramos que
	o algoritmo recebe como input duas priority queues de arcos (estradas num e aeroportos
	no outro) ordenadas segundo: 1) o peso dos arcos; 2) se o arco é um arco de aeroporto.

	Nas seguintes complexiades considera-se que $A$ = número de aeroportos, $R$ = número de estradas.

	\begin{enumerate}
		\item Inicialização 1: $O(V)$\\
			Criamos um vetor de booleanos que indica se cada um dos vértices já
			pertence à MST.
			Criamos ainda um vetor de Disjoint-Sets que vai ser utilizado no algortimo.
		\item Ciclo 1: $O(R\cdot log(V))$\\
			Executamos o ciclo principal do algoritmo de Kruskal, apenas em relação
			aos arcos de estradas.
			Sempre que escolhemos um arco, juntamo-lo à priority queue que contem os arcos de aeroporto.
			Ao longo da execução vamos computando o custo da MST; atualizado as
			flags do vetor de presença e contando o número de arcos já selecionados.
		\item Inicialização 2: $O(V)$\\
			Re-inicializamos o vetor dos Disjoint-Sets.
		\item Ciclo 2: $O( (V + A) \cdot log(V))$\\
			No segundo ciclo executamos novamente o algoritmo de Kruskal, mas apenas considerando
			os arcos escolhidos durante a ultima execução ($V-1$ no máximo) e os arcos de aeroporto (ainda não considerados).
			Esta solução funciona porque os restantes arcos já nunca seriam escolhidos.
			Os arcos que vão sendo escolhidos são colocados num vetor.
			Ao longo da execução vamos novamente atualizado as flags do vetor de presença
			e contando, separadamente, o número de estradas e aeroportos já selecionados.
		\item Finalização: $O(V)$\\
			No fim é necessário escolher qual é a solução ótima do problema


	\end{enumerate}


  \begin{center}
  \begin{tikzpicture}
    \begin{axis}[
        title={Execution time for problems with solution},
        xlabel={Number of photos [n]},
        ylabel={Time [s]},
        ymajorgrids=true,
        xmajorgrids=true,
        grid style=dashed
      ]

      \addplot [smooth, color=blue, mark=x] table [x=c, y=t] {./results1.data};
    \end{axis}
  \end{tikzpicture}
  \end{center}

  \begin{center}
  \begin{tikzpicture}
    \begin{axis}[
        title={Execution time for problems with solution},
        xlabel={Number of photos [n]},
        ylabel={Time / log(n)},
        ymajorgrids=true,
        xmajorgrids=true,
        grid style=dashed
      ]

      \addplot [only marks, color=blue, mark=x] table [x=c, y=tln] {./results1.data};
    \end{axis}
  \end{tikzpicture}
  \end{center}
\end{document}
