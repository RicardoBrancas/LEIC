\documentclass[a4paper]{article}
\usepackage[margin=3cm]{geometry}
\usepackage[portuguese]{babel}
\usepackage[utf8]{inputenc}
\usepackage[T1]{fontenc}
\usepackage{graphicx}
\usepackage{amsmath}
\usepackage{courier}
\usepackage{listings}
\usepackage{tikz}
\usepackage{tikz-qtree}

\lstset{language=SQL}
\lstset{basicstyle=\footnotesize\ttfamily,morekeywords={IF,REFERENCES,TYPE,ENUM,REPLACE,FUNCTION,RETURNS,VOID,DECLARE,BEGIN,DEFERRED,FOR,LOOP,LANGUAGE,RAISE,SETOF,WHILE,RETURN,QUERY,HASH,TEMP,BTREE}}

\begin{document}

    \begin{titlepage}
        \centering
        \includegraphics[width=0.5\textwidth]{IST_A_CMYK_POS.pdf}\par
        {\huge\bfseries Projeto de Bases de Dados, Parte 4\par}
        \vspace{2cm}
        {
        \Large
        \begin{tabular}{llll}
            83533 & Mihail Brinza & .\% & h \\
            83557 & Ricardo Brancas & .\% & h \\
            83883 & David Nunes & .\% & h
        \end{tabular}
        }
        \vfill
        \large
        Grupo 37, turno BD2251795L10 \par
        Professor Miguel Amaral

        \vspace{3cm}

        {\normalsize \today\par}
    \end{titlepage}

    \section{Restrições de Integridade}

    \section{Índices}

    \subsection{Índice 1}
    Para acelerar esta \textit{query} é necessário um índice que permita ordenação (como um índice
    \lstinline{btree}) na coluna \lstinline{fornecedor.nif} para que uma das colunas do join esteja
    indexada. No entanto esta coluna já uma \textit{primary key}, já possuindo um índice nestas condições.
%    Para acelerar da melhor forma esta \textit{query} são neceesários um \lstinline{btree} index
%    para a coluna \lstinline{fornecedor.nif} e outro para a coluna \lstinline{produto.forn_primario}
%    (de modo a permitir a execução de um \textit{merge join}).

    É ainda necessário um \lstinline{hash} index para a coluna \lstinline{produto.categoria}.
    Apresenta-se o código SQL correspondente:

    \begin{lstlisting}
-- CREATE INDEX fornecedor_nif_idx        ON fornecedor USING BTREE (nif);
-- CREATE INDEX produto_forn_primario_idx ON produto    USING BTREE (forn_primario);
CREATE INDEX produto_categoria_idx     ON produto    USING HASH  (categoria);
    \end{lstlisting}

    \subsection{Índice 2}
    É, pelos mesmos motivos que na alínea anterior, necessário existir um índice que permita a ordenação da coluna \lstinline{produto.ean}
    como, por exemplo, um índice \lstinline{btree}.
    No entanto como esta coluna já é uma \textit{primary key}, já existe um índice com estas
    características.

    \section{Modelo Multidimensional}

    \section{\textit{Data Analytics}}

\end{document}