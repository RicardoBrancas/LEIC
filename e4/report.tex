\documentclass[a4paper]{article}
\usepackage[margin=2.5cm]{geometry}
\usepackage[portuguese]{babel}
\usepackage[utf8]{inputenc}
\usepackage[T1]{fontenc}
\usepackage{graphicx}
\usepackage{amsmath}
\usepackage{courier}
\usepackage{listings}

\lstset{language=SQL}
\lstset{basicstyle=\footnotesize\ttfamily,showstringspaces=false,morekeywords={BEFORE,EACH,ROW,PROCEDURE,NEW,IF,REFERENCES,TYPE,ENUM,REPLACE,FUNCTION,RETURNS,VOID,DECLARE,BEGIN,DEFERRED,FOR,LOOP,LANGUAGE,RAISE,SETOF,WHILE,RETURN,QUERY,HASH,TEMP,BTREE}}

\begin{document}

    \begin{titlepage}
        \centering
        \includegraphics[width=0.5\textwidth]{IST_A_CMYK_POS.pdf}\par
        {\huge\bfseries Projeto de Bases de Dados, Parte 4\par}
        \vspace{2cm}
        {
        \Large
        \begin{tabular}{llll}
            83533 & Mihail Brinza & .\% & h \\
            83557 & Ricardo Brancas & .\% & h \\
            83883 & David Nunes & .\% & h
        \end{tabular}
        }
        \vfill
        \large
        Grupo 37, turno BD2251795L10 \par
        Professor Miguel Amaral

        \vspace{3cm}

        {\normalsize \today\par}
    \end{titlepage}

    \section{Restrições de Integridade}
    \lstinputlisting{ri.sql}

    \section{Índices}

    \subsection{Índice 1}\label{subsec:ind1}
    Esta \textit{query} contém um \lstinline{JOIN} cuja condição é de igualdade; a forma mais eficiente de o
    realizar é percorrendo uma das tabelas (ou um dos índices a ela associados) na totalidade e depois verificar
    (através de um índice de \lstinline{HASH}) se existe um registo correspondente na outra tabela.
    No entanto uma das tabelas (\lstinline{fornecedor}) já contém um índice sobre a coluna correta
    (\lstinline{produto}). Apesar de este índice ser do tipo \lstinline{BTREE}, muito provavelmente
    não compensa criar um novo índice do tipo \lstinline{HASH} (dependendo, no entento, da situação particular).

    Para acelerar a condição \lstinline{WHERE} devemos ainda criar um índice, também de \lstinline{HASH}, sobre
    a coluna \lstinline{produto.categoria}. Apresentamos os índices em questão:

    \begin{lstlisting}
-- CREATE INDEX fornecedor_nif_idx        ON fornecedor USING HASH (nif);
   CREATE INDEX produto_categoria_idx     ON produto    USING HASH  (categoria);
    \end{lstlisting}

    \subsection{Índice 2}
    Em relação à optimização do \lstinline{JOIN}, estamos numa situação idêntica à da secção~\ref{subsec:ind1}, não
    sendo necessário criar nenhum índice adicional (devido à existência da \textit{primary key}).


    Para otimizar o \lstinline{GROUP BY} seria últi criar um índice do tipo \lstinline{BTREE} sobre a coluna
    \lstinline{produto.ean} (ou \lstinline{fornece_sec.ean}) mas, mais uma vez, já existe um índice com essas
    características associado à \textit{primary key}. Como tal, não é necessário criar nenhum índice adicional
    para otimizar esta \textit{query}.

    \section{Modelo Multidimensional}
    \lstinputlisting{multidim.sql}

    \section{\textit{Data Analytics}}
    \lstinputlisting{dataAnalytic.sql}

\end{document}
